
%% \begin{document}


\begin{center}
  {\sc \bf LAWRENCE BERKELEY NATIONAL LABORATORY}\\
\end{center}
NERSC's newest supercomputer, named Cori, is a Cray XC40. Named for American biochemist Gerty Cori, the first American woman to win a Nobel Prize and the first woman to be so honored with the prize in Physiology or Medicine, Cori ranked as the 8th most powerful supercomputer in the world on the November 2017 list of Top 500 supercomputers in the world.   Cori is a unique among supercomputers of its size with two different kinds of nodes, 2,388 Intel Xeon "Haswell" processor nodes 9,688 Intel Xeon Phi "Knight's Landing" nodes. Cori also features a 1.8 PB Cray Data Warp Burst Buffer with I/O operating at a world's-best 1.7 TB/sec.
(source NERSC)
