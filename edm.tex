
Here are some of the extensions and evolutions being contemplated
for the event data model.  These are mainly aiming towards having
a more efficient representation, especially for GPUs and other
accelerators.

\begin{itemize}
\item The objects produced to be stored in AODs are in xAOD format,
  but many of the detector-specific and intermediate objects
  are not.  Some of these may need to be migrated to xAOD.

\item Currently, each xAOD variable is stored in a std::vector.  This means
  that each variable is stored in contiguous memory, but there is no relation
  between the allocations for different variables of the same object.
  It may be desirable to have more control over this; for example to be able
  to store all variables for a given object in a single contiguous
  block of memory.  This can be done transparently to the EDM clients,
  but would require changes to the xAOD static auxiliary store classes.
  There are likely also implications for persistency and backwards
  compatibility.

\item Many xAOD variables are simple types, but there are also some that
  use std::vector and ElementLink.  These, especially std::vector, are not
  so friendly for GPUs.  An alternate representation may be more
  efficient.  However, this would very likely affect client code.
  {\it \color{blue} Internal reconstruction EDM often not xAOD}

\item DataVector<T> acts like vector<T*>, and client code typically accesses
  the data through the pointers to the T instances.  However, this involves
  an extra pointer dereference; among other things, this prevents any
  vectorization.  This will also not be efficient for GPUs.

  From the start, it was imagined that variables could be accessed
  through the DataVector class directly, without requiring the additional
  dereference.  However, there has not been any support for this in the
  xAOD layer, and no client code currently uses this.  This should be
  addressed along with the update to interfaces discussed below.

\item The interfaces provided by DataVector and the xAOD classes
  are likely too complicated to be implemented as-is on GPUs and
  related architectures.  Some prototyping has been done on more
  streamlined interfaces.  However, a design should be settled
  on and integrated with the offine code in a compatible way,
  and in a way in which variable definitions do not need to be duplicated.

\item The EDM has always been accessible to python via PyROOT;
  however, the performance
  of this is not very good, since every access requires a call from
  python to C++.

  Prototypes exist of code that expose the underlying vectors used
  to hold xAOD variables to python more directly, with a class-like
  wrapper on top.  This is significantly more efficient;
  however, the current version loses much of the xAOD structure in so doing.
  It also does not interoperate directly with external python code
  that operates on numpy objects.

  It may be worth investigating if xAOD objects can be made available to python
  in a manner which is directly compatible with numpy and which also
  maintains the xAOD structure and interfaces.

\item In heterogeneous computing, one may need to deal with multiple
  representations of a data object; for example, one accessible from
  the CPU and one accessible from a GPU.  Some programming libraries
  try to hide this, but it may be worth considering if it would
  be useful to be able to represent alternate versions of the same
  object in the event store.

\item Many-core systems may also have a non-uniform memory architecture.
  Making efficient use of such architectures would require being able
  to better control memory allocations.
\end{itemize}
