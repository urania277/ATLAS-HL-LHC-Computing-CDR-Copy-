 
\label{sec:detdescr}

An accurate description of the ATLAS detector is crucial for the simulation and reconstruction of data. In preparation for Run 4, the current detector description (scheme and detectors), which has not changed in its driving principles since its inception 20 years ago, will need to be updated to modern standards and functionality to meet present and future requirements, especially to ensure maintainability and the smooth integration of Phase II upgrade detectors into the overall description. 



\subsection{GeoModel: Geometry kernel classes for detector description}

The current class library providing primitives for detector description is called GeoModel. It has been in service in ATLAS since 2003.  At the time of this writing, it depends only upon the Eigen matrix algebra library. It is supported by a heterogeneous set of database technologies and parsers. A new  effort in detector description, targeting Run-4, is now under  way. The goals include consolidating the various technologies in detector description, accurately describing the new  detectors in Run-4, and improving the development tools for detector  description. The defining objective is to dramatically shorten the detector description development cycle, by providing firstly, a means to achieve a immediate modification of geometry description; secondly, tools for immediate visual feedback; and thirdly, tools for rapid validation of geometries. These tools are largely independent of the Athena framework. 

The Run-4 ATLAS detector will be radically different from the runs-1, 2, and 3 detectors. The inner detector consisting of a Pixel detector, a silicon strip tracker, and a straw tracker, will be replaced by an all-silicon system called the Inner Tracker (ITK)\cite{Collaboration:2285585,Collaboration:2257755}. A silicon high-granularity timing detector\cite{Collaboration:2623663} will be installed in front of the endcap calorimeter face to provide vertex timing information to the reconstruction.  The remaining portion of the New Small Wheel\cite{Kawamoto:1552862} will be installed. A substantial portion of the barrel muon system will be replaced by new chambers to improve trigger and tracking performance.  Only the calorimeter systems remain largely intact; their upgrade program consists only of improvements to the readout electronics. 


\subsection{Future Work: the Streamlined Detector Description Workflow}

The aim for Run-4 is to have a unified GeoModel of the whole detector, steered by an unified XML-based database and parsed by a single software package. The description should be free of geometry clashes, optimized for a performant detector simulation, and free of unnecessary dependencies which impact the portability\footnote{We note here that a portable detector description may be the key to success in running simulation, at some future date, on novel platforms.}. We envision a streamlined workflow in which a developer creates or modifies a description of a piece of detector in which modifications to the geometry and visual feedback occur within a very short loop. The main portion of the work can occur on laptop computers with no local or remote installation of the ATLAS software stack. Instead, the environment consists of lightweight, platform-independent software.  The user modifies a local, XML-based file and/or a factory plugin that interprets this file.  The factory plugin creates a GeoModel description, and plugs into a visualization tool deriving from VP1~\cite{ref:vp1-web}, but tailored to geometry development and available outside of Athena, lxplus, Centos7, etc.  When the code is mature the factory is transferred to the ATLAS software stack and is invoked within Athena, as is presently the case.  At a later point in the initialization of Athena, extra information such as alignment information and readout geometry is layered upon the raw geometry. This tool suite is designed to allow turnaround times on the order of the Atlas SW development cycle (order 1 month), in order to rapidly integrate improvements to to the software description of the ATLAS detector. 

The programme of work includes the following items.  Firstly, the geometry kernel classes will be  reviewed, and the persistency model will be improved.  A uniform system for accessing XML databases for feeding data to the geometry builders will be developed and employed in future detector description code.  A visualization system (the "Geometry Explorer", \texttt{gmex}) is developed; it features extremely sophisticated visualization developed for the VP1 event display; together with the databases and the plugins, the system resembles a platform-independent CAD system for detector description. We foresee a tool suite consisting of automatic detection of geometry clashes and other anomalies, automatic generation of ``geantino'' maps, and auto-blending of volumes. Integration with Athena, including  interoperation with the alignment system, will be addressed. New detectors for Run-4 will be developed, and existing detector descriptions will be critically reviewed. Finally, all elements of the new infrastructure will be documented. 
Following this, and in some cases in parallel, existing detector description code will be reviewed, revised, and ported to the new system; the description of new detectors will be implemented using the new tools, and the geometry system will be put in a state of readiness for Run-4, and long-term maintenance. 

